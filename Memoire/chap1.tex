% Chapter 1

\chapter{INTRODUCTION GÉNÉRALE}
\label{ch:introduction_generale}


Dans le contexte actuel de la transformation digitale, les organisations doivent continuellement améliorer leur efficacité opérationnelle afin de rester compétitives. L’un des leviers essentiels de cette amélioration réside dans la modélisation et l’automatisation des processus métier. Les processus métier, définis comme l’ensemble des activités coordonnées permettant d’atteindre un objectif organisationnel, constituent le cœur du fonctionnement des entreprises modernes. Cependant, leur gestion pose encore de nombreux défis, notamment en termes de traçabilité, de flexibilité et d’optimisation.

Comment modéliser et automatiser efficacement les processus métier afin d’optimiser les performances organisationnelles, tout en garantissant une meilleure flexibilité, traçabilité et collaboration entre les différents acteurs ?  

Face à cette problématique, il devient nécessaire de concevoir et de mettre en œuvre une solution intégrée qui permette non seulement de représenter visuellement les processus à travers une notation standardisée comme BPMN (Business Process Model and Notation), mais aussi de les exécuter automatiquement à l’aide d’un moteur d’automatisation.  

L’objectif général de ce mémoire est donc de proposer une approche méthodologique et technique permettant de passer de la modélisation théorique des processus métier à leur automatisation effective dans un environnement informatique. Les objectifs spécifiques sont les suivants :
\begin{itemize}
    \item Étudier et analyser les concepts liés à la modélisation et l’automatisation des processus métier ;
    \item Concevoir une architecture logicielle adaptée intégrant un moteur d’exécution des processus ;
    \item Développer une application de modélisation et d’automatisation basée sur BPMN, Spring Boot et React ;
    \item Mettre en place un système de configuration et de suivi des tâches orienté utilisateurs ;
    \item Évaluer la solution proposée à travers un cas pratique et en analyser les performances.
\end{itemize}

Pour atteindre ces objectifs, une méthodologie mixte sera adoptée, combinant une revue de littérature pour identifier les meilleures pratiques existantes, une phase de conception et de développement logiciel selon une approche itérative, ainsi qu’une phase d’expérimentation et d’évaluation de la solution dans un scénario concret.

Ce mémoire est structuré en plusieurs chapitres :
\begin{itemize}
    \item Le chapitre 1 présente l’introduction générale du mémoire, y compris le contexte, la problématique, les objectifs, la méthodologie et le plan du mémoire.
    \item Le chapitre 2 présente une revue de la littérature sur la modélisation et l’automatisation des processus métier.
    \item Le chapitre 3 décrit l’architecture proposée ainsi que la conception de la solution.
    \item Le chapitre 4 détaille la méthodologie de mise en œuvre et le développement de l’application.
    \item Le chapitre 5 présente les résultats obtenus et une discussion sur leur portée.
    \item Enfin, le chapitre 6 conclut le mémoire en résumant les apports principaux et en proposant des perspectives d’amélioration et de recherche future.
\end{itemize}
